\documentclass[12pt]{article}

% --- Paquetes básicos ---
\usepackage[utf8]{inputenc}
\usepackage[spanish]{babel}
\usepackage{geometry}
\geometry{a4paper, margin=2.5cm}
\usepackage{graphicx}

\begin{document}

% --- 1. PORTADA REDISEÑADA ---
\begin{titlepage}
    \centering
    
    % Título de mayor peso
    \vspace*{3cm} 
    {\Huge \textbf{PROGRAMACION DE SERVICIOS}} \\ 
    \vspace{0.8cm}
    
    % Subtítulo (Actividad)
    {\huge \textit{Chat Grupal}} \\ 
    
    \vspace{2cm}
    \hrule % Línea divisoria
    \vspace{2cm}
    
    % Información detallada
    {\large
    \begin{tabular}{ll}
        \textbf{Estudiante:} & John Arenales Fernandez \\
        \textbf{Profesor:}   & Roberto Castro Liste \\
        \textbf{Año:}        & 2026 \\
    \end{tabular}
    }

    \vfill 
    
    {\large Aula Estudio}
    
\end{titlepage}

\clearpage % Salto a la siguiente página

% --- 2. ÍNDICE (Automático de LaTeX) ---
% Se coloca en su propia hoja como pediste originalmente
\tableofcontents 
\clearpage % Salto a la siguiente página

% --- 3. CONTENIDO DE EJEMPLO ---
% Aquí empieza el contenido real. Los números de página empezarán aquí.
\section{Introducción}

En la programación de servicios moderna, la comunicación entre aplicaciones es fundamental. Este proyecto implementa un \textbf{chat grupal en tiempo real} utilizando sockets y buffers, conceptos clave para entender cómo los sistemas se comunican a través de redes.

Un \textbf{socket} es un punto final de comunicación que actúa como intermediario entre dos programas que desean intercambiar información. Específicamente, utilizamos sockets TCP (Transmission Control Protocol), que garantizan una conexión confiable y ordenada entre el servidor y los clientes. Estos sockets utilizan \textbf{buffers} para almacenar temporalmente los datos antes de ser procesados, permitiendo que la información fluya eficientemente sin pérdida.

En esta práctica, implementamos una arquitectura cliente-servidor donde:

\begin{itemize}
    \item El \textbf{servidor} mantiene múltiples conexiones simultáneas con diferentes clientes
    \item Cada \textbf{cliente} se conecta al servidor a través de un socket dedicado
    \item Los \textbf{buffers} almacenan los mensajes entrantes y salientes
    \item Los mensajes se distribuyen a todos los clientes conectados (broadcast)
\end{itemize}

Este proyecto nos permite aplicar y consolidar los conceptos fundamentales vistos en clase sobre:

\begin{itemize}
    \item Comunicación por sockets (TCP/IP)
    \item Programación concurrente (manejo de múltiples hilos)
    \item Operaciones de entrada/salida en redes
    \item Sincronización entre procesos
    \item Desarrollo de aplicaciones cliente-servidor
\end{itemize}
\clearpage
 

\section{Descripcion del problema}
Como desarrolladores independientes, hemos sido contactados por una empresa del sector de las redes sociales tras identificar una limitación crítica en su plataforma: la falta de una interacción fluida entre sus usuarios. Actualmente, la aplicación carece de un servicio de mensajería interna, lo que impide la comunicación directa, ya sea de forma privada o grupal. El objetivo del proyecto es diseñar e implementar un sistema de mensajería instantánea integrado, que garantice la plena identificación de los interlocutores y permita a los usuarios personalizar su identidad mediante nombres de usuario de su elección.


\section{Conclusión}
Final del trabajo.

\end{document}