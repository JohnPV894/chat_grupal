\documentclass[12pt]{article}

% --- Paquetes básicos ---
\usepackage[utf8]{inputenc}
\usepackage[spanish]{babel}
\usepackage{geometry}
\geometry{a4paper, margin=2.5cm}
\usepackage{graphicx}

\begin{document}

% --- 1. PORTADA REDISEÑADA ---
\begin{titlepage}
    \centering
    
    % Título de mayor peso
    \vspace*{3cm} 
    {\Huge \textbf{PROGRAMACION DE SERVICIOS}} \\ 
    \vspace{0.8cm}
    
    % Subtítulo (Actividad)
    {\huge \textit{Chat Grupal}} \\ 
    
    \vspace{2cm}
    \hrule % Línea divisoria
    \vspace{2cm}
    
    % Información detallada
    {\large
    \begin{tabular}{ll}
        \textbf{Estudiante:} & John Arenales Fernandez \\
        \textbf{Profesor:}   & Roberto Castro Liste \\
        \textbf{Año:}        & 2026 \\
    \end{tabular}
    }

    \vfill 
    
    {\large Aula Estudio}
    
\end{titlepage}

\clearpage % Salto a la siguiente página

% --- 2. ÍNDICE (Automático de LaTeX) ---
% Se coloca en su propia hoja como pediste originalmente
\tableofcontents 
\clearpage % Salto a la siguiente página

% --- 3. CONTENIDO DE EJEMPLO ---
% Aquí empieza el contenido real. Los números de página empezarán aquí.
\section{Introducción}

En la programación de servicios moderna, la comunicación entre aplicaciones es fundamental. Este proyecto implementa un \textbf{chat grupal en tiempo real} utilizando sockets y buffers, conceptos clave para entender cómo los sistemas se comunican a través de redes.

Un \textbf{socket} es un punto final de comunicación que actúa como intermediario entre dos programas que desean intercambiar información. Específicamente, utilizamos sockets TCP (Transmission Control Protocol), que garantizan una conexión confiable y ordenada entre el servidor y los clientes. Estos sockets utilizan \textbf{buffers} para almacenar temporalmente los datos antes de ser procesados, permitiendo que la información fluya eficientemente sin pérdida.

En esta práctica, implementamos una arquitectura cliente-servidor donde:

\begin{itemize}
    \item El \textbf{servidor} mantiene múltiples conexiones simultáneas con diferentes clientes
    \item Cada \textbf{cliente} se conecta al servidor a través de un socket dedicado
    \item Los \textbf{buffers} almacenan los mensajes entrantes y salientes
    \item Los mensajes se distribuyen a todos los clientes conectados (broadcast)
\end{itemize}

Este proyecto nos permite aplicar y consolidar los conceptos fundamentales vistos en clase sobre:

\begin{itemize}
    \item Comunicación por sockets (TCP/IP)
    \item Programación concurrente (manejo de múltiples hilos)
    \item Operaciones de entrada/salida en redes
    \item Sincronización entre procesos
    \item Desarrollo de aplicaciones cliente-servidor
\end{itemize}
\clearpage
 

\section{Descripcion del problema}
Como desarrolladores independientes, hemos sido contactados por una empresa del sector de las redes sociales tras identificar una limitación crítica en su plataforma: la falta de una interacción fluida entre sus usuarios. Actualmente, la aplicación carece de un servicio de mensajería interna, lo que impide la comunicación directa, ya sea de forma privada o grupal. El objetivo del proyecto es diseñar e implementar un sistema de mensajería instantánea integrado, que garantice la plena identificación de los interlocutores y permita a los usuarios personalizar su identidad mediante nombres de usuario de su elección.
\\[0.7cm]
\begin{figure}[h!] 
    \centering
    \includegraphics[width=0.8\textwidth]{componentes_chat.jpg} \\[0.7cm]
    \caption{Componentes de un chat}
    \label{fig:identificacion_problema} % Sirve para referenciarla luego como: \ref{fig:identificacion_problema}
\end{figure}
\clearpage

\section{Requisitos}
Requisitos funcionales: \\
Definen las acciones y servicios que el sistema debe proporcionar. Especifican qué hace la aplicación, como las funcionalidades disponibles para el usuario, los procesos que realiza y la forma en que responde a determinadas acciones.
\\\\
Requisitos no funcionales:\\
Describen las cualidades y restricciones del sistema. Indican cómo debe funcionar la aplicación, incluyendo aspectos como el rendimiento, la seguridad, la usabilidad, la compatibilidad o la fiabilidad.
\\
\subsection{Requisitos funcionales}
\begin{itemize}
    \item Permitir a los usuarios registrarse con un nombre de usuario único.
    \item Facilitar el inicio de sesión para usuarios registrados.
    \item Opción de restablecer la contraseña a través del correo electrónico.
    \item Interfaz para enviar y recibir mensajes en tiempo real.
    \item Capacidad de crear y gestionar grupos de chat.
    \item Funcionalidad para enviar archivos y imágenes.
    \item Notificaciones de mensajes nuevos.
    \item Cierre de sesión seguro.
\end{itemize}

\subsection{Requisitos no funcionales}
\begin{itemize}
    \item Seguridad: Los datos del usuario y los mensajes deben estar cifrados.
    \item Escalabilidad: El sistema debe manejar un aumento en el número de usuarios sin degradar el rendimiento.
    \item Disponibilidad: El servicio debe estar operativo 24/7 con un tiempo de inactividad mínimo.
    \item Rendimiento: Los mensajes deben ser entregados en menos de un segundo.
    \item Usabilidad: La interfaz debe ser intuitiva y fácil de usar.
\end{itemize}
\clearpage

\section{Casos de uso}
\begin{itemize}
    \item \textbf{Registro de usuario:} Un nuevo usuario se registra proporcionando un nombre de usuario y una contraseña.
    \item \textbf{Inicio de sesión:} Un usuario existente ingresa con sus credenciales para acceder a su cuenta.
    \item \textbf{Restablecimiento de contraseña:} Un usuario que olvidó su contraseña solicita un restablecimiento a través de su correo electrónico.
    \item \textbf{Envío de mensajes:} Un usuario envía un mensaje a otro usuario o a un grupo.
    \item \textbf{Recepción de mensajes:} Un usuario recibe un mensaje de otro usuario o de un grupo.
    \item \textbf{Creación de grupos:} Un usuario crea un nuevo grupo de chat e invita a otros usuarios a unirse.
    \item \textbf{Gestión de perfil:} Un usuario actualiza su información personal o cambia su contraseña.
    \item \textbf{Cierre de sesión:} Un usuario cierra sesión de su cuenta de forma segura.
\end{itemize}
\clearpage
\section{Stack Tecnologico}
\subsection{Github}
Plataforma para el control de versiones del proyecto, permitiendo un seguimiento detallado de los cambios y la colaboración entre desarrolladores.
\subsection{Figma}
Herramienta de diseño colaborativo utilizada para crear prototipos de la interfaz de usuario y definir la experiencia visual de la aplicación.
\subsection{Java}
Lenguaje de programación principal utilizado en el desarrollo del backend de la aplicación, garantizando robustez y escalabilidad.
\subsection{JavaFx}
Framework utilizado para construir la interfaz gráfica de usuario (GUI) en aplicaciones de escritorio, proporcionando una experiencia de usuario rica y interactiva.
\subsection{IntelliJ IDEA}
Entorno de desarrollo integrado (IDE) utilizado para escribir, depurar y ejecutar el código fuente en Java, mejorando la productividad del desarrollador.
\subsection{Visual Studio Code}
Editor de código fuente utilizado para editar archivos de configuración y otros lenguajes de programación, ofreciendo flexibilidad y ligereza en el desarrollo.

\clearpage
\end{document}